%%\documentstyle[twocolumn,psfig]{article}
\documentstyle[11pt,psfig]{book}
%%\topmargin=-.32in
%%\headsep=.4in
%%\textheight=8.5in
%%\textwidth=6in
%%\oddsidemargin=.15in
%%\evensidemargin=.25in
\pagestyle{empty}
\begin{document}

\def\roadrunner{{\em Roadrunner}}
\def\rrfs{{\tt rrfs}}


\chapter{Bootstrap}

Bootstrapping is the initialization sequence required to begin
nominal kernel operations.  This same process is used whether
the power on the hardware is cycled or a soft reset occurs.

Bootstrapping occurs in two stages.  The first stage boot program
executes in x86 Real Mode and the second stage boot program
executes in x86 Protected Mode.  The first and second stage boot
programs are located in the first 16 sectors of the \roadrunner\ 
file system from which the computer boots.  The first sector
contains the first stage boot program and the remaining 15 sectors
contain the second stage boot program.

Data that describes system parameters is passed through the
bootstrap to the kernel.  The first stage boot program passes
information to the second stage boot program which then fowards
the information to the kernel.


\section{First Stage}

The first stage is responsible for loading the second stage boot
program into memory.  When the computer is started, either at power
on or after a reset, the first stage boot program is loaded at
location 07c0:0000 in real memory.  The first stage boot program
displays a simple boot message:

\begin{verbatim}
Roadrunner boot
\end{verbatim}

\noindent The first stage boot program then copies itself to
location 9000:0000 in real memory, zeros the memory starting at
location 0000:2000 in real memory in which the second stage boot
program will be loaded, stores the BIOS view of the hard disk
geometry at location 0000:1000 in real memory, and finally begins
to load the second stage boot program.

Loading the second stage is dependent on whether the bootstrap
is occuring from a floppy disk or a hard disk.  In both cases,
the second stage boot program is stored in the 15 sectors
immediately following the boot sector.  For a floppy based
bootstrap, these sectors are located immediately after the boot
sector.  For a hard disk based bootstrap, these sectors are
located in the partition from which the bootstrap is occuring.
The first stage boot program transfers control to the second
stage by jumping to the first location in the second stage
boot program memory.


\section{Second Stage}

The second stage is responsible for loading the operating system
kernel into memory.  When control is handed to the second stage,
the x86 CPU is still in Real Mode and the second stage program is
executing from memory starting at 0000:2000.  The second stage
first shifts the CPU to Protected Mode.  The use of Protected Mode
for kernel loading allows the second stage program to load the
kernel directly into memory above 1 Mbyte (20-bit Real Mode
addressing is limited to 1 Mbyte).  However, Protected Mode does
not allow the use of BIOS functions to accomplish loading from
secondary storage devices.  The second stage boot program includes
minimal device drivers for various secondary storage devices from
which kernel loading can take place.

After switching to Protected Mode, various hardware-oriented tasks
are performed, including setting the mode of the A20 address line,
initializing an Interrupt Descriptor Table, clearing the display.
and copying of the boot parameters passed in from the first
stage boot program.  The boot parameters are copied because the
memory at location 0x1000 (the Protected Mode equivalent of
0000:1000) is used to buffer data read from secondary storage
during the kernel load.  Once this copy is complete, loading
of the kernel commences.

Kernel loading is dependent on what secondary storage device is
being used.  The kernel is assumed to be present in the root
directory of a \roadrunner\  file system on whatever secondary
storage device the computer booted from.  The second stage boot
program loads the Master Boot Record (MBR), File Allocation
Table (FAT), and the root directory.  These data are used to
locate the kernel file.  The kernel file contains an Executable
and Linkable Format (ELF) executable program.  The program is
assume to be linked for location 0x100000 (1 Mbyte).  The second
stage boot program loads the kernel file section by section,
placing only loadable sections into their proper locations in
physical memory.  During the load, the second stage boot program
displays information about the kernel file sections in the
following format:

\begin{verbatim}
text=0x274d0 data=0x1628+0x3600
\end{verbatim}

\noindent This line indicates the size of the kernel file text,
data, and bss (zeroed data) areas in memory.  The data and bss
section sizes are displayed as two summed parameters.

Once loading is complete, the saved copy of the boot parameters
provided by the first stage boot program is copied back to location
0x1000 for use by the kernel.  The kernel assumes, similar to the
second stage boot program, that the boot parameters are located
at 0x1000.  The second stage boot program starts the kernel by
jumping to the start address indicated in the kernel file,
typically location 0x100000.

\chapter{Memory Management}

Memory management encompasses two functions, allocation and
protection.  Allocation refers to the mechanisms for granting
regions of memory for use by applications.  Protection refers
to the mechanisms used to prevent the access by one thread to
memory allocated to another thread or the kernel.  Memory
allocation is handled by tracking requests to allocate and
free memory.  Protection is provided using page-based Memory
Management Unit (MMU) hardware.

There are three basic memory management data structures used
in \roadrunner.  Each of the data structures in the following
list is described in greater detail in following sections.

\begin{enumerate}
\item {\em Region Table:} an array of regions that track the
allocations of all the physical memory in the system

\item {\em Page Tables:} each process belongs to a protection
domain that is defined by the page tables that are associated
with that process

\item {\em Kernel Map:} a list of mappings that are entered
into all the page tables in the system but are accessible
only when the CPU is in supervisor mode
\end{enumerate}

\noindent The first data structure is used for allocation.
The second two data structures are used to implement protection.
All three data structures are used by the kernel memory
management routines that are exported for use by the rest
of the kernel subsystems and by user applications through
the system call interface.

Figure \ref{memmgmt} illustrates these three data structures
in in use.  The Region Table is not shown here.  Rather, the
{\tt freelist} and {\tt alloclist} pointers reference lists
that partition the Region Table.  {\tt kmaptab[]} is a table of
Kernel Map entries.  The Page Tables are shown for one process.

\begin{center}\begin{figure}[h]
\centerline{\psfig{figure=fig/memmgmt.eps,height=3.75in}}
\caption{\label{memmgmt} \roadrunner\  Memory Management}
\end{figure}\end{center}


\section{Allocation}

The basic unit of memory allocation is the {\em region}.
A region is defined as a page-aligned, contiguous sequence
of addressable locations that are tracked by a starting address
and a length that must be a multiple of the page size.  The
entire physical memory on a given machine is managed using
a boundary-tag heap implementation in \roadrunner.  Figure
\ref{region} illustrates the basic data structure used to
track the allocations of memory regions.  Each region is
tracked using its starting address, {\tt start}, and length,
{\tt len}.  Each region is owned by the process that allocated
it originally, or by a process to which ownership has been
transferred after allocation.  The {\tt proc} field tracks
which process currently owns the region.  Two double-linked
lists of region data structures are maintained, using the
{\tt prev} and {\tt next} fields, each in ascending order
of starting address.  The first is the free list, those
regions that are not allocated to any process.  The second
is the allocated list, those regions that are being used
by some process.

\begin{center}\begin{figure}[h]
\centerline{\psfig{figure=fig/region.eps,height=1.25in}}
\caption{\label{region} A region on one of the heap lists}
\end{figure}\end{center}

Table \ref{regionroutines} lists the routines used to manage
the heap.  The {\tt valid\_region()} routine provides a check
for whether a pointer, specified by the {\tt start} parameter,
corresponds to the starting address of some region.  It also
serves as a general lookup routine for locating a region data
structure given a starting address.  The rest of the routines
take a pointer to a region data structure like the one
illustrated in Figure \ref{region} as their first parameter.
The {\tt region\_clear()} routine sets the fields of a region
data structure to initialized values.  The {\tt region\_insert()}
routine inserts a region in ascending starting address order
into a double-linked region list, specified by the {\tt list}
parameter.  This routine is used to insert a region into either
the free or allocated region lists.  The {\tt region\_remove()}
routine removes a region from the specified {\tt list}.  The
{\tt region\_split()} routine takes one region and splits it
into two regions.  The {\tt size} parameter specifies the
offset from the beginning of the original region where the
split is to occur.

\begin{table}[h]
\caption{\label{regionroutines} Region management routines}
\begin{tabbing}
\hspace{0.25in} \= \kill
{\tt valid\_region(start)} \\
\> Check whether pointer corresponds to the starting address of a region \\

{\tt region\_clear(region)} \\
\> Initialize a region data structure \\

{\tt region\_insert(region, list)} \\
\> Insert a region into a double-linked region list \\

{\tt region\_remove(region, list)} \\
\> Remove a region from a region list \\

{\tt region\_split(region, size)} \\
\> Split a region into two regions \\
\end{tabbing} \end{table}


\section{Protection}

The basic memory protection design is based on logical
(or virtual) addresses being mapped one-to-one to physical
addresses.  This means that the value of logical address
is the same as its corresponding physical address and that
all processes reside in the same logical as well as physical
address space.

Protection is based on {\em domains}.  A domain is a set of
memory pages that are mapped using a set of page tables.
In addition, a set of memory pages associated with the operating
system kernel called the {\em kernel map} is kept.  The kernel
map is mapped into all domains but is accessible only in
supervisor mode.

Table \ref{vmroutines} lists the routines in the \roadrunner\ 
kernel that implement basic memory protection operations.

\begin{table}[h]
\caption{\label{vmroutines} Page table management routines}
\begin{tabbing}
\hspace{0.25in} \= \kill
{\tt vm\_map(pt, page, attr)} \\
\> Map a single page into a set of page tables \\

{\tt vm\_map\_range(pt, start, len, attr)} \\
\> Map a sequence of contigous pages into a set of page tables \\

{\tt vm\_unmap(pt, page)} \\
\> Remove the mapping for a single page from a set of page tables \\

{\tt vm\_unmap\_range(pt, start, len)} \\
\> Remove the mapping for a contiguous sequence of pages from a set \\
\> of page tables \\

{\tt vm\_kmap\_insert(entry)} \\
\> Insert a sequence of pages into the kernel map \\

{\tt vm\_kmap\_remove(entry)} \\
\> Remove a sequence of pages from the kernel map \\

{\tt vm\_kmap(pt)} \\
\> Update specified page tables to reflect the current mappings in the \\
\> kernel map \\
\end{tabbing} \end{table}


\subsection{Page Tables}

The kernel keeps track of the page tables present in the
system by maintaining a list of page table records.  Figure
\ref{ptrec} illustrates the page table record data structure
and the associated page tables.  The page table records are
kept in a single-linked list using the {\tt next} field.
If multiple threads are executed within a single protection
domain, the {\tt refcnt} field tracks the total number of
threads within the domain.  The {\tt pd} field points to
the actual page table data structures.  Note that there
is a single pointer to the page tables themselves.  This
design implies that the page tables are arranged contiguously
in memory.  An assessment of current MMU implementations in
several popular CPU architectures indicates that this is
a reasonable assumption.  More details on the page table
structures of several popular processor architectures are
given in Section 4.

\begin{center}\begin{figure}[h]
\centerline{\psfig{figure=fig/ptrec.eps,height=1.25in}}
\caption{\label{ptrec} A page table record and its associated
page tables}
\end{figure}\end{center}

The first four routines in Table \ref{vmroutines} implement
the basic protection mechanism.  They enter and remove address
mappings to and from page tables, respectively.  All four of
these routines operate on a set of page tables specified by
their first parameter, {\tt pt}.  The {\tt vm\_map()} routine
provides the fundamental operation of inserting a mapping for
a single page into a set of page tables.  The page found at
the location specified by the {\tt start} parameter is inserted
with the protection attributes specified by the {\tt attr}
parameter into the specified page tables.  {\tt vm\_map\_range()}
is provided for convenience as a front-end to {\tt vm\_map()}
to allow mapping a sequence of contiguous pages with a single
call.  The {\tt start} parameter specifies the address of the
first of a contiguous sequence of pages.  The {\tt len}
parameter specifies the length, in bytes, of the page sequence.
The initial implementation or the {\tt vm\_map\_range()}
routine makes calls to {\tt vm\_map()} for each page in the
specified range.  This implementation is obviously ripe for
optimization.

The {\tt vm\_unmap()} routine balances {\tt vm\_map()} by
providing the removal of a single page mapping from a page
table.  The {\tt page} parameter specifies the starting
address of the page that is to be unmapped.
{\tt vm\_unmap\_range()} is provided as a front-end to
{\tt vm\_unmap()} to allow removal of a sequence of contiguous
entries with a single call.  {\tt start} specifies the starting
address of the page sequence and {\tt len} gives the byte
length of the page sequence to be unmapped.  The
{\tt vm\_unmap\_range()} routine also make individual calls
to {\tt vm\_unmap()} for each page in the specified range
and can also be optimized.


\subsection{The Kernel Map}

In some virtual memory system designs that provide
separate address spaces, the kernel has been maintained
in its own address space.  In the \roadrunner\  system,
the memory used to hold the kernel and its associated data
structures are mapped into all the page tables in the
system.  Kernel memory protection is provided by making
these pages accessible only when the CPU has entered
supervisor mode and that happens only when an interrupt
occurs or a system call is made.  The result is that
system calls require only a transition from user to
supervisor mode rather than a full context switch.

The kernel map is an array of kernel map entries where
each entry represents a region that is entered in the
kernel map.  Figure \ref{kmap} illustrates the structure
of one of these kernel map entries and the region of
memory that it represents.  The {\tt start} and {\tt len}
fields track the starting address and length of the region.
The {\tt attr} field stores the attributes that are
associated with the pages in the region.  This information
is used when the pages are entered into a set of page
tables by the {\tt vm\_kmap()} routine.

\begin{center}\begin{figure}[h]
\centerline{\psfig{figure=fig/kmap.eps,height=1.25in}}
\caption{\label{kmap} A kernel map entry and its associated
memory region}
\end{figure}\end{center}

The last three routines in  Table \ref{vmroutines} provide
the API for managing the kernel map.  The
{\tt vm\_kmap\_insert()} routine enters a kernel map entry,
specified by the {\tt entry} parameter, into the kernel map.
The {\tt vm\_kmap\_remove()} routine removes a previously
entered kernel map entry, also specified by the {\tt entry}
parameter, from the kernel map.  The {\tt vm\_kmap()} routine
causes a set of page tables, specified by the {\tt pt}
parameter, to be updated with the current kernel map entries.


\subsection{Page Faults}

The most important function of page faults in a system
using separate virtual address spaces is demand paging.
Demand paging of user code can also be done using this
approach under two additional conditions.  First, all of
the physical memory required to hold the program must be
allocated when the program is started.  Second, code
relocation needs to be performed on-the-fly when sections
of the program were loaded on-demand.

If demand paging of user code is implemented, page fault
handling is similar to systems where separate virtual
address spaces are used.  When a page fault occurs, the
appropriate kernel service determines whether the fault
occured due to a code reference and if so, it loads the
appropriated section of code and restarts the faulting
process.

The initial \roadrunner\  implementation does not
currently support demand paging of program code.  As
such, page fault handling is trivial, resulting in the
termination of the process that caused the fault.


\section{Kernel Memory Managment}

Table \ref{kmem} lists the routines that are used by
kernel subsystems and by applications through the system
call interface to allocate and free memory from the
global heap.

\begin{table}[h]
\caption{\label{kmem} Kernel memory management routines}
\begin{tabbing}
\hspace{0.25in} \= \kill
{\tt malloc(size)} \\
\> Allocate a region of memory in the calling process's protection \\
\> domain \\

{\tt free(start)} \\
\> Free a region of memory previously allocated to the calling process \\

{\tt kmalloc(size)} \\
\> Allocate a region of memory for the kernel \\

{\tt kfree(start)} \\
\> Free a region of memory previously allocated to the kernel \\
\end{tabbing} \end{table}

The {\tt malloc()} routine performs an allocation on behalf
of a process by performing a first-fit search of the free
list.  When a region is found that is at least as large as
a request specified by the {\tt size} parameter, it is
removed from the free list using the {\tt region\_remove()}
routine.  The remainder is split off using the
{\tt region\_split()} routine and returned to the free
list using the {\tt region\_insert()} routine.  The
region satisfying the request is then mapped into the
protection domain of the calling process using the
{\tt vm\_map\_range()} routine.

The {\tt free()} routine returns a previously allocated
region to the heap.  After obtaining the region
corresponding to the specified {\tt start} parameter
using the {\tt valid\_region()} lookup, the
{\tt region\_insert()} routine is used to enter the
region into the free list.  The inserted region is
then merged with its neighbors, both previous and next
if they are adjacent.  Adjacency means that the two
regions together form a contiguous sequence of pages.
Merging is done to reduce fragmentation.

The {\tt kmalloc()} routine allocates some memory on
behalf of the kernel.  After obtaining a region from
the heap in a manner similar to the {\tt malloc()} routine
based on the specified {\tt size} request, an entry is
placed into the kernel map using the {\tt vm\_kmap\_insert()}
routine.  This action records the new region as an element
of the kernel map.  Subsequent calls to {\tt vm\_kmap()}
will cause the new region to be accessible as part of the
kernel when a process is running in supervisor mode.

The {\tt kfree()} routine first removes the kernel mapping
for the region specified by the {\tt start} parameter using
{\tt vm\_kmap\_remove()}.  The region is then placed back
on the free list using {\tt region\_insert()}.

\chapter{Interrupts}

Interrupts are signals received from hardware elements that
require immediate attention.  Interrupt management includes
configuration of the physical hardware that generates interrupts
and management of the procedures, called interrupt service
routines, that are called in response to interrupts.


\section{Interrupt Controllers}

A standard x86 based Personal Computer can receive interrupt
from two sources.  The first is a set of exceptions or traps
that are generated internally by the CPU.  The second source
is two (2) Intel 8259 Programmable Interrupt Controllers (PICs).
These PIC devices are cascaded together to allow up to 15
external Interrupt Requests (IRQs) from various hardware devices.
Newer PCs can also contain an Advanced Programmable Interrupt
Controller (APIC).  \roadrunner\  does not currently support
these newer devices.

\roadrunner\  abstracts all interrupts into a set of interrupt
numbers ranging from 0 to 63.  Figure \ref{irq} illustrates
the structure of interrupt generation on IBM PC compatible
computers.  The first 32 \roadrunner\ interrupts are mapped
onto the processor traps and exceptions.  The next 16
interrupts are mapped onto the two cascaded PICs.  The
interrupts between 48 and 63 are reserved for use by software
interrupts.  The only software interrupt currently defined is
the system call specified at interrupt number 48.

\begin{center}\begin{figure}[h]
\centerline{\psfig{figure=fig/irq.eps,height=3.75in}}
\caption{\label{irq} IBM PC Compatible Interrupt Hardware}
\end{figure}\end{center}

Software drivers tend to refer to these interrupts with
respect to hardware elements that generate them.  Since the
processor traps and exceptions are occupy the low end of the
\roadrunner\  interrupt numbering scheme, they map directly.
IRQs are often referred to by their IRQ number.  The macro 
{\tt IRQ2INTR} is provided to translate an IRQ number to
a \roadrunner\  interrupt number.

The processor exceptions are generated by various conditions
that occur withing the CPU during operation.  They cannot
be masked in general.  The hardware IRQs can be masked and
unmasked individually at any time.


\section{Interrupt Service Routines}

Interrupt Service Routines (ISRs) are the first piece of code
(after some operating system kernel bookkeeping) to be executed
when an interrupt occurs.  \roadrunner\  allows ISRs to be
managed dynamically at runtime.



\chapter{Time}

The kernel tracks the progress of time.  The kernel uses a timer
interrupt to indicate the passage of time and keeps the time of
day for use by the kernel and application programs.  The kernel
also implements time-slicing of the CPU to implement
multi-processing.


\section{Time of Day}


\section{Time-Slicing}


\chapter{Processes}

Processes represent executing application programs.  Each process
requires a collection of hardware and kernel resources to execute
an application program.  The \roadrunner\  kernel implements
fully preemptible multi-processing, i.e. any process can be
interrupted at any time.


\section{Process States}


\section{Process Queues}


\chapter{Mutexes}

Mutexes are used within the kernel to synchronize simultaneous
accesses by multiple processes to certain hardware and kernel
elements.  Processes that want to access a shared resource may
be forced to wait in a process queue.


\chapter{Events}

Events are analogous to interrupts.  The difference is that
handling of events is done using processes rather than interrupt
service routines.  Processes are placed in a process list and
all are scheduled for execution when a specified event occurs.


\chapter{Device Drivers}

Input/Output (I/O) is handled by peripheral device hardware.
Device drivers represent the kernel elements that manage I/O to
and from hardware devices.  The kernel provides a uniform view of
all hardware devices, despite their differing I/O characteristics.


\chapter{File Systems}

There are various file system formats in which data can be
stored and presented.  The kernel has the ability to use multiple
file system format simultaneously within a single file system
name space.  A generic Application Programming Interface (API)
is provided to access data regardless of the underlying file
system format.


\section{The System File System ({\tt sysfs})}


\section{The Device File System ({\tt devfs})}


\section{The \roadrunner\  File System (\rrfs)}

The specific file system format used by the \roadrunner\ 
kernel as its primary data storage is the \roadrunner\ 
file system (\rrfs).  The \rrfs\  is a 32-bit
File-Allocation Table (FAT) based file system.  The main
goal of \rrfs\  design is implementation simplicity.


\subsection{Path Name Conventions}

\rrfs\  file names are limited to 48 characters.  There
is no separation of the file name into a name and extension.
File names are case-sensitive, upper and lower case are
preserved internally.  The file name character set is:

\begin{verbatim}
A-Z a-z 0-9 _ . - % ( ) + , $ # : ; < = > ? @ [ ] ^ ! { | } ~
\end{verbatim}


\subsection{\rrfs\  Organization}

The \rrfs\  organization is similar to DOS FAT file systems.
In fact, a DOS FAT16 implementation was completed first and
the \rrfs\  design was then initiated to bypass some of it's
deficiencies.  Microsoft's FAT32 file system also served as
an influence but was not strictly adhered to since backward
compatibility was not a priority.


\subsubsection{Partition Layout}

Figure \ref{rrfspart} illustrates the layout of a disk
partition that contains an {\tt rrfs} file system.  The
Master Boot Record (MBR) contains the first stage boot
program and some global parameters that describe the file
system.  The {\tt rrfs} MBR is illustrated in Figure
\ref{rrfsmbr}.  The boot loader is responsible for loading
and starting execution of the operating system.  The
operating system kernel is assumed to reside in the root
directory and must be named {\tt kernel}.  The file
allocation table (FAT) contains pointers to all of the
data clusters in the partition.  Each FAT entry is an
unsigned 32-bit integer.  Each FAT contains entries for
all clusters, including the first root directory cluster.
The only cluster that has a known location in the partition
is the first cluster of the root directory, which is assumed
to reside in the first data cluster of the partition.  The
size of the root directory is not limited.

\begin{table}[t] \begin{center}
\begin{tabular}{|l|l|} \hline
Sectors & Description \\ \hline \hline
1 & Master boot record \\ \hline
15 & Boot loader (second stage) \\ \hline
??? & File allocation table (FAT) \\ \hline
??? & File allocation table copy (FAT) \\ \hline
??? & First root directory cluster \\ \hline
??? & Data clusters \\ \hline
\end{tabular}
\caption{\label{rrfspart} {\tt rrfs} partition layout}
\end{center} \end{table}

\begin{table}[t] \begin{center}
\begin{tabular}{|l|l|l|} \hline
Size & Offset & Field \\ \hline \hline
4 & 0 & Jump: {\tt EB XX 90 90} \\ \hline
4 & 4 & Tracks (or cylinders) \\ \hline
4 & 8 & Heads \\ \hline
4 & 12 & Sectors \\ \hline
2 & 14 & Bytes per sector \\ \hline
4 & 18 & Total sectors \\ \hline
2 & 20 & Boot sectors \\ \hline
4 & 24 & Fat sectors \\ \hline
2 & 26 & Sectors per cluster \\ \hline
4 & 30 & Data clusters \\ \hline
\end{tabular}
\caption{\label{rrfsmbr} {\tt rrfs} master boot record}
\end{center} \end{table}


\subsubsection{Directory Entry}

Each {\tt rrfs} directory entry has a total length of 64
bytes.  This allows 8 directory entries in a 512 byte disk
sector.  Figure \ref{rrfsde} illustrates the layout of
a directory entry.  The name field holds the file name
character string and is case-sensitive.  The size field
holds the file size in an unsigned 32-bit integer.  The
start cluster field holds the first cluster of the file
in an unsigned 32-bit integer.

The attributes field is an unsigned 16-bit integer whose
individual bits are used as indicators.  The time field
is a sequence of 3 bytes.  Each byte is interpreted as
an unsigned 8-bit integer.  The values for each of these
bytes is given in Figure \ref{rrfsdetime}.  The date
field is a sequence of 3 bytes.  Each byte is interpreted
as an unsigned 8-bit integer.  The values for each of
these bytes is given in Figure \ref{rrfsdedate}.

\begin{table}[t] \begin{center}
\begin{tabular}{|l|l|l|} \hline
Size & Offset & Field \\ \hline \hline
48 & 0 & Name \\ \hline
2 & 48 & Attributes \\ \hline
3 & 50 & Time \\ \hline
3 & 53 & Date \\ \hline
4 & 56 & Size \\ \hline
4 & 60 & Start cluster \\ \hline
\end{tabular}
\caption{\label{rrfsde} {\tt rrfs} directory entry}
\end{center} \end{table}

\begin{table}[t] \begin{center}
\begin{tabular}{|l|l|} \hline
Offset & Field \\ \hline \hline
0 & Hour (0-23) \\ \hline
1 & Minute (0-59) \\ \hline
2 & Second (0-59) \\ \hline
\end{tabular}
\caption{\label{rrfsdetime} {\tt rrfs} directory entry time format}
\end{center} \end{table}

\begin{table}[t] \begin{center}
\begin{tabular}{|l|l|} \hline
Offset & Field \\ \hline \hline
0 & Month (0-11) \\ \hline
1 & Day of month \\ \hline
2 & Year (+1900) \\ \hline
\end{tabular}
\caption{\label{rrfsdedate} {\tt rrfs} directory entry date format}
\end{center} \end{table}



\chapter{Lexical Analysis}

Lexical analysis provides capabilities for complex ASCII string
manipulation.  Capabilities are provided to decompose a string
into a sequence of arguments, to provide argument typing, and
to convert number strings into numeric values.


\chapter{Pathnames}

Files represent a generic mechanism for storing and presenting
data.  Files are stored in a hierarchical file system and
pathnames are used to locate files in a file system.  Routines
are provided that allow construction and decomposition of
pathnames.


\chapter{Files}

File represent the basic mechanism for storing and presenting
data.  The kernel implements a POSIX interface for access to
all file data.


\chapter{Program Loading}

Programs are loaded via the file system interface into memory
for execution.  \roadrunner\  uses the Executable and Linkable
Format (ELF) executable file format for application programs.


\chapter{System Calls}

System calls are used by application programs or libraries to
invoke kernel services.  The \roadrunner\  system call interface
is based on the POSIX standards.


\end{document}
