\chapter{Interrupts}

Interrupts are signals received from hardware elements that
require immediate attention.  Interrupt management includes
configuration of the physical hardware that generates interrupts
and management of the procedures, called interrupt service
routines, that are called in response to interrupts.


\section{Interrupt Controllers}

A standard x86 based Personal Computer can receive interrupt
from two sources.  The first is a set of exceptions or traps
that are generated internally by the CPU.  The second source
is two (2) Intel 8259 Programmable Interrupt Controllers (PICs).
These PIC devices are cascaded together to allow up to 15
external Interrupt Requests (IRQs) from various hardware devices.
Newer PCs can also contain an Advanced Programmable Interrupt
Controller (APIC).  \roadrunner\  does not currently support
these newer devices.

\roadrunner\  abstracts all interrupts into a set of interrupt
numbers ranging from 0 to 63.  Figure \ref{irq} illustrates
the structure of interrupt generation on IBM PC compatible
computers.  The first 32 \roadrunner\ interrupts are mapped
onto the processor traps and exceptions.  The next 16
interrupts are mapped onto the two cascaded PICs.  The
interrupts between 48 and 63 are reserved for use by software
interrupts.  The only software interrupt currently defined is
the system call specified at interrupt number 48.

\begin{center}\begin{figure}[h]
\centerline{\psfig{figure=fig/irq.eps,height=3.75in}}
\caption{\label{irq} IBM PC Compatible Interrupt Hardware}
\end{figure}\end{center}

Software drivers tend to refer to these interrupts with
respect to hardware elements that generate them.  Since the
processor traps and exceptions are occupy the low end of the
\roadrunner\  interrupt numbering scheme, they map directly.
IRQs are often referred to by their IRQ number.  The macro 
{\tt IRQ2INTR} is provided to translate an IRQ number to
a \roadrunner\  interrupt number.

The processor exceptions are generated by various conditions
that occur withing the CPU during operation.  They cannot
be masked in general.  The hardware IRQs can be masked and
unmasked individually at any time.


\section{Interrupt Service Routines}

Interrupt Service Routines (ISRs) are the first piece of code
(after some operating system kernel bookkeeping) to be executed
when an interrupt occurs.  \roadrunner\  allows ISRs to be
managed dynamically at runtime.
