\chapter{Memory Management}

Memory management encompasses two functions, allocation and
protection.  Allocation refers to the mechanisms for granting
regions of memory for use by applications.  Protection refers
to the mechanisms used to prevent the access by one thread to
memory allocated to another thread or the kernel.  Memory
allocation is handled by tracking requests to allocate and
free memory.  Protection is provided using page-based Memory
Management Unit (MMU) hardware.

There are three basic memory management data structures used
in \roadrunner.  Each of the data structures in the following
list is described in greater detail in following sections.

\begin{enumerate}
\item {\em Region Table:} an array of regions that track the
allocations of all the physical memory in the system

\item {\em Page Tables:} each process belongs to a protection
domain that is defined by the page tables that are associated
with that process

\item {\em Kernel Map:} a list of mappings that are entered
into all the page tables in the system but are accessible
only when the CPU is in supervisor mode
\end{enumerate}

\noindent The first data structure is used for allocation.
The second two data structures are used to implement protection.
All three data structures are used by the kernel memory
management routines that are exported for use by the rest
of the kernel subsystems and by user applications through
the system call interface.

Figure \ref{memmgmt} illustrates these three data structures
in in use.  The Region Table is not shown here.  Rather, the
{\tt freelist} and {\tt alloclist} pointers reference lists
that partition the Region Table.  {\tt kmaptab[]} is a table of
Kernel Map entries.  The Page Tables are shown for one process.

\begin{center}\begin{figure}[h]
\centerline{\psfig{figure=fig/memmgmt.eps,height=3.75in}}
\caption{\label{memmgmt} \roadrunner\  Memory Management}
\end{figure}\end{center}


\section{Allocation}

The basic unit of memory allocation is the {\em region}.
A region is defined as a page-aligned, contiguous sequence
of addressable locations that are tracked by a starting address
and a length that must be a multiple of the page size.  The
entire physical memory on a given machine is managed using
a boundary-tag heap implementation in \roadrunner.  Figure
\ref{region} illustrates the basic data structure used to
track the allocations of memory regions.  Each region is
tracked using its starting address, {\tt start}, and length,
{\tt len}.  Each region is owned by the process that allocated
it originally, or by a process to which ownership has been
transferred after allocation.  The {\tt proc} field tracks
which process currently owns the region.  Two double-linked
lists of region data structures are maintained, using the
{\tt prev} and {\tt next} fields, each in ascending order
of starting address.  The first is the free list, those
regions that are not allocated to any process.  The second
is the allocated list, those regions that are being used
by some process.

\begin{center}\begin{figure}[h]
\centerline{\psfig{figure=fig/region.eps,height=1.25in}}
\caption{\label{region} A region on one of the heap lists}
\end{figure}\end{center}

Table \ref{regionroutines} lists the routines used to manage
the heap.  The {\tt valid\_region()} routine provides a check
for whether a pointer, specified by the {\tt start} parameter,
corresponds to the starting address of some region.  It also
serves as a general lookup routine for locating a region data
structure given a starting address.  The rest of the routines
take a pointer to a region data structure like the one
illustrated in Figure \ref{region} as their first parameter.
The {\tt region\_clear()} routine sets the fields of a region
data structure to initialized values.  The {\tt region\_insert()}
routine inserts a region in ascending starting address order
into a double-linked region list, specified by the {\tt list}
parameter.  This routine is used to insert a region into either
the free or allocated region lists.  The {\tt region\_remove()}
routine removes a region from the specified {\tt list}.  The
{\tt region\_split()} routine takes one region and splits it
into two regions.  The {\tt size} parameter specifies the
offset from the beginning of the original region where the
split is to occur.

\begin{table}[h]
\caption{\label{regionroutines} Region management routines}
\begin{tabbing}
\hspace{0.25in} \= \kill
{\tt valid\_region(start)} \\
\> Check whether pointer corresponds to the starting address of a region \\

{\tt region\_clear(region)} \\
\> Initialize a region data structure \\

{\tt region\_insert(region, list)} \\
\> Insert a region into a double-linked region list \\

{\tt region\_remove(region, list)} \\
\> Remove a region from a region list \\

{\tt region\_split(region, size)} \\
\> Split a region into two regions \\
\end{tabbing} \end{table}


\section{Protection}

The basic memory protection design is based on logical
(or virtual) addresses being mapped one-to-one to physical
addresses.  This means that the value of logical address
is the same as its corresponding physical address and that
all processes reside in the same logical as well as physical
address space.

Protection is based on {\em domains}.  A domain is a set of
memory pages that are mapped using a set of page tables.
In addition, a set of memory pages associated with the operating
system kernel called the {\em kernel map} is kept.  The kernel
map is mapped into all domains but is accessible only in
supervisor mode.

Table \ref{vmroutines} lists the routines in the \roadrunner\ 
kernel that implement basic memory protection operations.

\begin{table}[h]
\caption{\label{vmroutines} Page table management routines}
\begin{tabbing}
\hspace{0.25in} \= \kill
{\tt vm\_map(pt, page, attr)} \\
\> Map a single page into a set of page tables \\

{\tt vm\_map\_range(pt, start, len, attr)} \\
\> Map a sequence of contigous pages into a set of page tables \\

{\tt vm\_unmap(pt, page)} \\
\> Remove the mapping for a single page from a set of page tables \\

{\tt vm\_unmap\_range(pt, start, len)} \\
\> Remove the mapping for a contiguous sequence of pages from a set \\
\> of page tables \\

{\tt vm\_kmap\_insert(entry)} \\
\> Insert a sequence of pages into the kernel map \\

{\tt vm\_kmap\_remove(entry)} \\
\> Remove a sequence of pages from the kernel map \\

{\tt vm\_kmap(pt)} \\
\> Update specified page tables to reflect the current mappings in the \\
\> kernel map \\
\end{tabbing} \end{table}


\subsection{Page Tables}

The kernel keeps track of the page tables present in the
system by maintaining a list of page table records.  Figure
\ref{ptrec} illustrates the page table record data structure
and the associated page tables.  The page table records are
kept in a single-linked list using the {\tt next} field.
If multiple threads are executed within a single protection
domain, the {\tt refcnt} field tracks the total number of
threads within the domain.  The {\tt pd} field points to
the actual page table data structures.  Note that there
is a single pointer to the page tables themselves.  This
design implies that the page tables are arranged contiguously
in memory.  An assessment of current MMU implementations in
several popular CPU architectures indicates that this is
a reasonable assumption.  More details on the page table
structures of several popular processor architectures are
given in Section 4.

\begin{center}\begin{figure}[h]
\centerline{\psfig{figure=fig/ptrec.eps,height=1.25in}}
\caption{\label{ptrec} A page table record and its associated
page tables}
\end{figure}\end{center}

The first four routines in Table \ref{vmroutines} implement
the basic protection mechanism.  They enter and remove address
mappings to and from page tables, respectively.  All four of
these routines operate on a set of page tables specified by
their first parameter, {\tt pt}.  The {\tt vm\_map()} routine
provides the fundamental operation of inserting a mapping for
a single page into a set of page tables.  The page found at
the location specified by the {\tt start} parameter is inserted
with the protection attributes specified by the {\tt attr}
parameter into the specified page tables.  {\tt vm\_map\_range()}
is provided for convenience as a front-end to {\tt vm\_map()}
to allow mapping a sequence of contiguous pages with a single
call.  The {\tt start} parameter specifies the address of the
first of a contiguous sequence of pages.  The {\tt len}
parameter specifies the length, in bytes, of the page sequence.
The initial implementation or the {\tt vm\_map\_range()}
routine makes calls to {\tt vm\_map()} for each page in the
specified range.  This implementation is obviously ripe for
optimization.

The {\tt vm\_unmap()} routine balances {\tt vm\_map()} by
providing the removal of a single page mapping from a page
table.  The {\tt page} parameter specifies the starting
address of the page that is to be unmapped.
{\tt vm\_unmap\_range()} is provided as a front-end to
{\tt vm\_unmap()} to allow removal of a sequence of contiguous
entries with a single call.  {\tt start} specifies the starting
address of the page sequence and {\tt len} gives the byte
length of the page sequence to be unmapped.  The
{\tt vm\_unmap\_range()} routine also make individual calls
to {\tt vm\_unmap()} for each page in the specified range
and can also be optimized.


\subsection{The Kernel Map}

In some virtual memory system designs that provide
separate address spaces, the kernel has been maintained
in its own address space.  In the \roadrunner\  system,
the memory used to hold the kernel and its associated data
structures are mapped into all the page tables in the
system.  Kernel memory protection is provided by making
these pages accessible only when the CPU has entered
supervisor mode and that happens only when an interrupt
occurs or a system call is made.  The result is that
system calls require only a transition from user to
supervisor mode rather than a full context switch.

The kernel map is an array of kernel map entries where
each entry represents a region that is entered in the
kernel map.  Figure \ref{kmap} illustrates the structure
of one of these kernel map entries and the region of
memory that it represents.  The {\tt start} and {\tt len}
fields track the starting address and length of the region.
The {\tt attr} field stores the attributes that are
associated with the pages in the region.  This information
is used when the pages are entered into a set of page
tables by the {\tt vm\_kmap()} routine.

\begin{center}\begin{figure}[h]
\centerline{\psfig{figure=fig/kmap.eps,height=1.25in}}
\caption{\label{kmap} A kernel map entry and its associated
memory region}
\end{figure}\end{center}

The last three routines in  Table \ref{vmroutines} provide
the API for managing the kernel map.  The
{\tt vm\_kmap\_insert()} routine enters a kernel map entry,
specified by the {\tt entry} parameter, into the kernel map.
The {\tt vm\_kmap\_remove()} routine removes a previously
entered kernel map entry, also specified by the {\tt entry}
parameter, from the kernel map.  The {\tt vm\_kmap()} routine
causes a set of page tables, specified by the {\tt pt}
parameter, to be updated with the current kernel map entries.


\subsection{Page Faults}

The most important function of page faults in a system
using separate virtual address spaces is demand paging.
Demand paging of user code can also be done using this
approach under two additional conditions.  First, all of
the physical memory required to hold the program must be
allocated when the program is started.  Second, code
relocation needs to be performed on-the-fly when sections
of the program were loaded on-demand.

If demand paging of user code is implemented, page fault
handling is similar to systems where separate virtual
address spaces are used.  When a page fault occurs, the
appropriate kernel service determines whether the fault
occured due to a code reference and if so, it loads the
appropriated section of code and restarts the faulting
process.

The initial \roadrunner\  implementation does not
currently support demand paging of program code.  As
such, page fault handling is trivial, resulting in the
termination of the process that caused the fault.


\section{Kernel Memory Managment}

Table \ref{kmem} lists the routines that are used by
kernel subsystems and by applications through the system
call interface to allocate and free memory from the
global heap.

\begin{table}[h]
\caption{\label{kmem} Kernel memory management routines}
\begin{tabbing}
\hspace{0.25in} \= \kill
{\tt malloc(size)} \\
\> Allocate a region of memory in the calling process's protection \\
\> domain \\

{\tt free(start)} \\
\> Free a region of memory previously allocated to the calling process \\

{\tt kmalloc(size)} \\
\> Allocate a region of memory for the kernel \\

{\tt kfree(start)} \\
\> Free a region of memory previously allocated to the kernel \\
\end{tabbing} \end{table}

The {\tt malloc()} routine performs an allocation on behalf
of a process by performing a first-fit search of the free
list.  When a region is found that is at least as large as
a request specified by the {\tt size} parameter, it is
removed from the free list using the {\tt region\_remove()}
routine.  The remainder is split off using the
{\tt region\_split()} routine and returned to the free
list using the {\tt region\_insert()} routine.  The
region satisfying the request is then mapped into the
protection domain of the calling process using the
{\tt vm\_map\_range()} routine.

The {\tt free()} routine returns a previously allocated
region to the heap.  After obtaining the region
corresponding to the specified {\tt start} parameter
using the {\tt valid\_region()} lookup, the
{\tt region\_insert()} routine is used to enter the
region into the free list.  The inserted region is
then merged with its neighbors, both previous and next
if they are adjacent.  Adjacency means that the two
regions together form a contiguous sequence of pages.
Merging is done to reduce fragmentation.

The {\tt kmalloc()} routine allocates some memory on
behalf of the kernel.  After obtaining a region from
the heap in a manner similar to the {\tt malloc()} routine
based on the specified {\tt size} request, an entry is
placed into the kernel map using the {\tt vm\_kmap\_insert()}
routine.  This action records the new region as an element
of the kernel map.  Subsequent calls to {\tt vm\_kmap()}
will cause the new region to be accessible as part of the
kernel when a process is running in supervisor mode.

The {\tt kfree()} routine first removes the kernel mapping
for the region specified by the {\tt start} parameter using
{\tt vm\_kmap\_remove()}.  The region is then placed back
on the free list using {\tt region\_insert()}.
