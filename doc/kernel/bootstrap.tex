\chapter{Bootstrap}

Bootstrapping is the initialization sequence required to begin
nominal kernel operations.  This same process is used whether
the power on the hardware is cycled or a soft reset occurs.

Bootstrapping occurs in two stages.  The first stage boot program
executes in x86 Real Mode and the second stage boot program
executes in x86 Protected Mode.  The first and second stage boot
programs are located in the first 16 sectors of the \roadrunner\ 
file system from which the computer boots.  The first sector
contains the first stage boot program and the remaining 15 sectors
contain the second stage boot program.

Data that describes system parameters is passed through the
bootstrap to the kernel.  The first stage boot program passes
information to the second stage boot program which then fowards
the information to the kernel.


\section{First Stage}

The first stage is responsible for loading the second stage boot
program into memory.  When the computer is started, either at power
on or after a reset, the first stage boot program is loaded at
location 07c0:0000 in real memory.  The first stage boot program
displays a simple boot message:

\begin{verbatim}
Roadrunner boot
\end{verbatim}

\noindent The first stage boot program then copies itself to
location 9000:0000 in real memory, zeros the memory starting at
location 0000:2000 in real memory in which the second stage boot
program will be loaded, stores the BIOS view of the hard disk
geometry at location 0000:1000 in real memory, and finally begins
to load the second stage boot program.

Loading the second stage is dependent on whether the bootstrap
is occuring from a floppy disk or a hard disk.  In both cases,
the second stage boot program is stored in the 15 sectors
immediately following the boot sector.  For a floppy based
bootstrap, these sectors are located immediately after the boot
sector.  For a hard disk based bootstrap, these sectors are
located in the partition from which the bootstrap is occuring.
The first stage boot program transfers control to the second
stage by jumping to the first location in the second stage
boot program memory.


\section{Second Stage}

The second stage is responsible for loading the operating system
kernel into memory.  When control is handed to the second stage,
the x86 CPU is still in Real Mode and the second stage program is
executing from memory starting at 0000:2000.  The second stage
first shifts the CPU to Protected Mode.  The use of Protected Mode
for kernel loading allows the second stage program to load the
kernel directly into memory above 1 Mbyte (20-bit Real Mode
addressing is limited to 1 Mbyte).  However, Protected Mode does
not allow the use of BIOS functions to accomplish loading from
secondary storage devices.  The second stage boot program includes
minimal device drivers for various secondary storage devices from
which kernel loading can take place.

After switching to Protected Mode, various hardware-oriented tasks
are performed, including setting the mode of the A20 address line,
initializing an Interrupt Descriptor Table, clearing the display.
and copying of the boot parameters passed in from the first
stage boot program.  The boot parameters are copied because the
memory at location 0x1000 (the Protected Mode equivalent of
0000:1000) is used to buffer data read from secondary storage
during the kernel load.  Once this copy is complete, loading
of the kernel commences.

Kernel loading is dependent on what secondary storage device is
being used.  The kernel is assumed to be present in the root
directory of a \roadrunner\  file system on whatever secondary
storage device the computer booted from.  The second stage boot
program loads the Master Boot Record (MBR), File Allocation
Table (FAT), and the root directory.  These data are used to
locate the kernel file.  The kernel file contains an Executable
and Linkable Format (ELF) executable program.  The program is
assume to be linked for location 0x100000 (1 Mbyte).  The second
stage boot program loads the kernel file section by section,
placing only loadable sections into their proper locations in
physical memory.  During the load, the second stage boot program
displays information about the kernel file sections in the
following format:

\begin{verbatim}
text=0x274d0 data=0x1628+0x3600
\end{verbatim}

\noindent This line indicates the size of the kernel file text,
data, and bss (zeroed data) areas in memory.  The data and bss
section sizes are displayed as two summed parameters.

Once loading is complete, the saved copy of the boot parameters
provided by the first stage boot program is copied back to location
0x1000 for use by the kernel.  The kernel assumes, similar to the
second stage boot program, that the boot parameters are located
at 0x1000.  The second stage boot program starts the kernel by
jumping to the start address indicated in the kernel file,
typically location 0x100000.
