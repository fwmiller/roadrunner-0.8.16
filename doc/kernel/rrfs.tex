\section{The \roadrunner\  File System (\rrfs)}

The specific file system format used by the \roadrunner\ 
kernel as its primary data storage is the \roadrunner\ 
file system (\rrfs).  The \rrfs\  is a 32-bit
File-Allocation Table (FAT) based file system.  The main
goal of \rrfs\  design is implementation simplicity.


\subsection{Path Name Conventions}

\rrfs\  file names are limited to 48 characters.  There
is no separation of the file name into a name and extension.
File names are case-sensitive, upper and lower case are
preserved internally.  The file name character set is:

\begin{verbatim}
A-Z a-z 0-9 _ . - % ( ) + , $ # : ; < = > ? @ [ ] ^ ! { | } ~
\end{verbatim}


\subsection{\rrfs\  Organization}

The \rrfs\  organization is similar to DOS FAT file systems.
In fact, a DOS FAT16 implementation was completed first and
the \rrfs\  design was then initiated to bypass some of it's
deficiencies.  Microsoft's FAT32 file system also served as
an influence but was not strictly adhered to since backward
compatibility was not a priority.


\subsubsection{Partition Layout}

Figure \ref{rrfspart} illustrates the layout of a disk
partition that contains an {\tt rrfs} file system.  The
Master Boot Record (MBR) contains the first stage boot
program and some global parameters that describe the file
system.  The {\tt rrfs} MBR is illustrated in Figure
\ref{rrfsmbr}.  The boot loader is responsible for loading
and starting execution of the operating system.  The
operating system kernel is assumed to reside in the root
directory and must be named {\tt kernel}.  The file
allocation table (FAT) contains pointers to all of the
data clusters in the partition.  Each FAT entry is an
unsigned 32-bit integer.  Each FAT contains entries for
all clusters, including the first root directory cluster.
The only cluster that has a known location in the partition
is the first cluster of the root directory, which is assumed
to reside in the first data cluster of the partition.  The
size of the root directory is not limited.

\begin{table}[t] \begin{center}
\begin{tabular}{|l|l|} \hline
Sectors & Description \\ \hline \hline
1 & Master boot record \\ \hline
15 & Boot loader (second stage) \\ \hline
??? & File allocation table (FAT) \\ \hline
??? & File allocation table copy (FAT) \\ \hline
??? & First root directory cluster \\ \hline
??? & Data clusters \\ \hline
\end{tabular}
\caption{\label{rrfspart} {\tt rrfs} partition layout}
\end{center} \end{table}

\begin{table}[t] \begin{center}
\begin{tabular}{|l|l|l|} \hline
Size & Offset & Field \\ \hline \hline
4 & 0 & Jump: {\tt EB XX 90 90} \\ \hline
4 & 4 & Tracks (or cylinders) \\ \hline
4 & 8 & Heads \\ \hline
4 & 12 & Sectors \\ \hline
2 & 14 & Bytes per sector \\ \hline
4 & 18 & Total sectors \\ \hline
2 & 20 & Boot sectors \\ \hline
4 & 24 & Fat sectors \\ \hline
2 & 26 & Sectors per cluster \\ \hline
4 & 30 & Data clusters \\ \hline
\end{tabular}
\caption{\label{rrfsmbr} {\tt rrfs} master boot record}
\end{center} \end{table}


\subsubsection{Directory Entry}

Each {\tt rrfs} directory entry has a total length of 64
bytes.  This allows 8 directory entries in a 512 byte disk
sector.  Figure \ref{rrfsde} illustrates the layout of
a directory entry.  The name field holds the file name
character string and is case-sensitive.  The size field
holds the file size in an unsigned 32-bit integer.  The
start cluster field holds the first cluster of the file
in an unsigned 32-bit integer.

The attributes field is an unsigned 16-bit integer whose
individual bits are used as indicators.  The time field
is a sequence of 3 bytes.  Each byte is interpreted as
an unsigned 8-bit integer.  The values for each of these
bytes is given in Figure \ref{rrfsdetime}.  The date
field is a sequence of 3 bytes.  Each byte is interpreted
as an unsigned 8-bit integer.  The values for each of
these bytes is given in Figure \ref{rrfsdedate}.

\begin{table}[t] \begin{center}
\begin{tabular}{|l|l|l|} \hline
Size & Offset & Field \\ \hline \hline
48 & 0 & Name \\ \hline
2 & 48 & Attributes \\ \hline
3 & 50 & Time \\ \hline
3 & 53 & Date \\ \hline
4 & 56 & Size \\ \hline
4 & 60 & Start cluster \\ \hline
\end{tabular}
\caption{\label{rrfsde} {\tt rrfs} directory entry}
\end{center} \end{table}

\begin{table}[t] \begin{center}
\begin{tabular}{|l|l|} \hline
Offset & Field \\ \hline \hline
0 & Hour (0-23) \\ \hline
1 & Minute (0-59) \\ \hline
2 & Second (0-59) \\ \hline
\end{tabular}
\caption{\label{rrfsdetime} {\tt rrfs} directory entry time format}
\end{center} \end{table}

\begin{table}[t] \begin{center}
\begin{tabular}{|l|l|} \hline
Offset & Field \\ \hline \hline
0 & Month (0-11) \\ \hline
1 & Day of month \\ \hline
2 & Year (+1900) \\ \hline
\end{tabular}
\caption{\label{rrfsdedate} {\tt rrfs} directory entry date format}
\end{center} \end{table}
