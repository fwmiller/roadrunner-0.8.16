\documentstyle{article}
\pagestyle{empty}
\flushbottom
\begin{document}

\def\roadrunner{{\em Roadrunner/pk}}

\title{\roadrunner\  Installation}
\date{}
\maketitle
\thispagestyle{empty}


There are several ways to run the \roadrunner\  operating system.  The
system can be run as a program on another operating system using an
emulator, e.g. bochs ({\tt http://www.bochs.com}) or VMware
({\tt http://www.vmware.com}).  The system can be built to boot
standalone off a single floppy disk and the system can be installed
and booted off an IDE hard disk.

The first step is to extract and build the source code distribution.  A
FreeBSD or Linux host can be used for this purpose.  The source code
distribution is usually contained in a tarball with a name like,
{\tt roadrunner.tgz}.  To extract the distribution, execute the following
command:

\begin{verbatim}
% tar xzvf roadrunner.tgz
\end{verbatim}

\noindent The result is a source code distribution tree rooted by a
directory named {\tt roadrunner}.

\subsection*{Building for Emulation}

To build the system for use with an emulator, a file is generated that
is equivalent to a floppy disk image.  This file can be used in place of
a floppy disk to boot the emulator.  To create the image file, execute
the following commands:

\begin{verbatim}
% cd roadrunner
% make
\end{verbatim}

\noindent The result is a file called {\tt image} in the {\tt roadrunner}
directory.  This image file can be used to boot the emulator.  It can also
be transferred to a 3 1/2'' 1.44 Mbyte floppy disk using a command such
as:

\begin{verbatim}
% dd if=roadrunner/image of=/dev/fd0
\end{verbatim}

\noindent Once this command is complete, the floppy disk can be used to
boot a computer into \roadrunner.  The use of {\tt dd} copies an image
that is the entire floppy disk, all 1.44 Mbytes of it.  To build a bootable
floppy disk, it is faster to follow the procedure in the next section,
which copies only the elements needed to the floppy disk.


\subsection*{Building a Bootable Floppy Disk}

To build the system on a 3 1/2'' 1.44 Mbyte floppy disk, insert a floppy
disk in the floppy disk drive and execute the following commands:

\begin{verbatim}
% cd roadrunner
% make floppy
\end{verbatim}

\noindent The result is a bootable \roadrunner\  floppy disk.


\subsection*{Installation on a Hard Disk}

Before describing the procedure for installing the system on a hard
disk, the customary warning is appropriate.  {\em The procedure has
the potential to destroy all the data on you hard disk.  Backup all your
data before attempting it.  Cornfed Systems is not responsible for any
loss incurred by following these instructions.}

The procedure used to install the system on an IDE hard disk uses the
bootable floppy disk we built in the last section.  {\em The current
release of \roadrunner\  allows installation on the primary master
IDE disk only.}  To install the \roadrunner\  system on the hard disk,
execute the following procedure:

\begin{enumerate}
\item Boot the computer using the \roadrunner\  floppy disk.

\item It may be necessary to partition the hard drive.  Use the
{\tt fdisk} program to do this by executing the following command:

\begin{verbatim}
/> fdisk
\end{verbatim}

\noindent A quick tutorial on the \roadrunner\  {\tt fdisk} program is
provided at the end of this document.

\item The partition into which the system is to be installed must be
formatted with the {\tt rrfs} file system.  To format the first partition
on the primary master IDE hard disk, execute the following command:

\begin{verbatim}
/> format /dev/wd0a -b /boot
\end{verbatim}

\noindent This command will install a bootable file system into the
partition only.  If the hard disk is to be shared with other operating
systems, this is the appropriate command.  If \roadrunner\  will be the
only operating system on the hard disk, a boot program must be written
to the Master Boot Record (MBR) as well.  This can be accomplished by
adding the {\tt -mbr} option to the format command.  The remaining
partitions on the primary master IDE hard disk can be addressed using
{\tt wd0b}, {\tt wd0c}, and {\tt wd0d} in place of {\tt wd0a}.

\item The remaining steps will copy the kernel and executable command
programs to the new file system.  Before these copies can take place
the new file system must be mounted.  Execute the following command:

\begin{verbatim}
/> mount wd0a /mnt
\end{verbatim}

\noindent The new file system is mounted at the {\tt /mnt} mount point.

\item A series of commands are now executed to place necessary files in
their proper locations on the hard disk.  Execute the following sequence
of commands:

\begin{verbatim}
/> cp /kernel /mnt
/> mkdir /mnt/bin
/> cp /bin/* /mnt/bin
/> mkdir /mnt/man
/> cp /man/* /mnt/man
\end{verbatim}

\noindent The sequence will copy the kernel to the root directory of the
new file system and then create a {\tt bin} directory on the new file
system and copy several command programs into the directory.  The
sequence also creates a {\tt man} directory on the new file system
and copies the man pages into it.

\item At this point the file system on the hard disk is capable of booting
\roadrunner.  All that is left is to unmount the file system, eject the
floppy disk, and boot into \roadrunner\  off the hard drive by performing
the following commands:

\begin{verbatim}
/> umount /mnt
/> reboot
\end{verbatim}

\end{enumerate}


\subsection*{Appendix: Using \roadrunner\  {\tt fdisk}}

When the {\tt fdisk} program starts up, you are presented with the
current contents of the partition table in tabular form.  The following
example shows this table for a fresh, unpartitioned hard drive.

{\small \begin{verbatim}
     ------start------ -------end-------
part track head sector track head sector   offset       size type
   0     0    0      0     0    0      0        0          0 unused
   1     0    0      0     0    0      0        0          0 unused
   2     0    0      0     0    0      0        0          0 unused
   3     0    0      0     0    0      0        0          0 unused
(a)dd (b)oot (t)ype (c)ommit (d)elete (p)rint e(x)it
fdisk>
\end{verbatim}}

The add command will create a new partition, subject to a number of
geometrical constraints.  The boot command will mark a partition as
active, i.e. the partition to boot from at startup time.  The commit
command writes the contents of the partition table to the MBR on the
disk.  The delete command will remove a partition.  The print command
dumps the table in the form given in the example.  The exit command
terminates the {\tt fdisk} program.

Here's and example creation of an 8 Mbyte bootable {\tt rrfs} partition
in the first slot of the partition table:

{\small \begin{verbatim}
fdisk> a
enter partition number (0-3)? 0
partition size (in Kbytes)? 8192
no next partition
no previous partition
add partition 0 offset 63 size 16443

     ------start------ -------end-------
part track head sector track head sector   offset       size type
   0     0    1      1   130    1     63       63      16443 rrfs
   1     0    0      0     0    0      0        0          0 unused
   2     0    0      0     0    0      0        0          0 unused
   3     0    0      0     0    0      0        0          0 unused
(a)dd (b)oot (t)ype (c)ommit (d)elete (p)rint e(x)it
fdisk> b
enter partition number (0-3)? 0

     ------start------ -------end-------
part track head sector track head sector   offset       size type
*  0     0    1      1   130    1     63       63      16443 rrfs
   1     0    0      0     0    0      0        0          0 unused
   2     0    0      0     0    0      0        0          0 unused
   3     0    0      0     0    0      0        0          0 unused
(a)dd (b)oot (t)ype (c)ommit (d)elete (p)rint e(x)it
fdisk> c
write partition table

ARE YOU SURE (uppercase 'Y' to confirm)? Y
(a)dd (b)oot (t)ype (c)ommit (d)elete (p)rint e(x)it
fdisk> e
/>
\end{verbatim}}

The {\tt a} command adds a new partition.  You are prompted for the
partition number and the desired size.  {\tt fdisk} will try to fit
the size request in based on the constraints presented by other
partitions already present.  You will be informed if the size request
cannot be implemented.
The {\tt b} command sets the partition as the active partition that
will be booted from.
The {\tt c} command writes the in-memory copy of the partition table
to the disk.  No changes to the partition table are made permanent
until a commit is performed.  You are prompted for confirmation.  An
uppercase `{\tt Y}' is the only response that results in the partition
table being written to disk.
The {\tt d} command deletes a partition from the table.
The {\tt p} command simply prints contents of the current
in-memory partition table.
The {\tt t} assigns a file system type for a partition.  The type
is specified as a hex value in the range, $0$ to $ff$.
The {\tt x} command exits {\tt fdisk}.

Partition 0 corresponds to {\tt wd0a} in the \roadrunner\  device naming
scheme.  Parition 1 corresponds {\tt wd0b}, partition 2 corresponds to
{\tt wd0c}, and partition 3 corresponds to {\tt wd0d}.


\end{document}
